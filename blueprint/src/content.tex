% In this file you should put the actual content of the blueprint.
% It will be used both by the web and the print version.
% It should *not* include the \begin{document}
%
% If you want to split the blueprint content into several files then
% the current file can be a simple sequence of \input. Otherwise It
% can start with a \section or \chapter for instance.

Halbach says: ``I will not clearly distinguish between theories and the systems that generate them,'' but perhaps we should. The problem is that for Halbach's purposes only a system's axioms matter, but perhaps we need the system's inference rules as well; I'm not sure. My preference goes out to the word 'system' as we should explicitly discuss proofs when talking about conservativity. Theorems, being ``a set of formulae closed under first-order logical consequence,'' (Halbach, p. 29) do not contain any information what rules govern sentence generation in proofs. Or are we allowed to define a formal proof simply as ``a finite sequence of sentences, each of which is an axiom of $\mathcal{S}$ or from the preceding sentences according to the rules of first-order inference''? I don't know, but something feels imprecise about the phrase 'first-order inference'.

\begin{definition}[$\mathcal{L}_T$]
    \label{def:LT}
    We define the language $\mathcal{L}_T$ as the language resulting from adding the predicate symbol $T$ to the language $\mathcal{L}$ of \texttt{PA}. 
\end{definition}

\begin{definition}[formal proof]
    \label{def:formal-proof}
    We define a formal proof in a logical system $\mathcal{S}$ as a finite sequence of sentences, each of which is an axiom of $\mathcal{S}$ or follows from the preceding sentences according to $\mathcal{S}$'s rules of inference (Wikipedia: Formal proof).
\end{definition}

\begin{definition}[\texttt{PAT}]
    \label{def:PAT}
    \uses{def:LT}
    We define the system \texttt{PAT} as the system of Peano arithmetic formulated in $\mathcal{L}_T$ including the induction schema for each formula of the language $\mathcal{L}_T$.
\end{definition}

\begin{definition}[\texttt{TB}]
    \label{def:TB}
    \uses{def:PAT}
    The system \texttt{TB} comprises all axioms of \texttt{PAT}. Moreover all sentences of the form $T\ulcorner\varphi\urcorner$ are axioms of the system where $\varphi$ is a sentence of the language of $\mathcal{L}$.
\end{definition}

\begin{definition}[conservativity]
    \label{def:cons}
    \uses{def:LT}
    A truth system $...$ [\textcolor{red}{how do you get this symbol?}] in the language $\mathcal{L}_T$ is conservative over a system $\mathcal{S}$ formulated in language $\mathcal{L}$ without the truth predicate if and only if all $...$-theorems $\varphi$ in the language $\mathcal{L}$ are also theorems of $\mathcal{S}$.
\end{definition}

\begin{lemma}[finite axioms in TB]
    \label{lem:finit-ax-tb}
    \uses{def:TB}
    In a proof in \texttt{TB} only finitely many axioms can occur.
\end{lemma}

\begin{proof}
    \uses{def:formal-proof} 
    Let $p$ be a proof in \texttt{TB}. Then, by Definition \ref{def:formal-proof}, $p$ is a finite sequence of sentences each of which is an axiom of \texttt{TB} or follows from the preceding sentences according to first-order rules of inference. 
    By Definition \ref{def:formal-proof} we have that 
    
    As TB is a proof system we have this immediately by Definition \ref{def:formal-proof}.
\end{proof}

\begin{theorem}
    \label{thm:tb-cons}
    % \lean{[the right reference]}
    \texttt{TB} is conservative over \texttt{PA}.
\end{theorem}

\begin{proof} \uses{def:cons, def:TB}  We show how to transform any given \texttt{TB}-proof of a formula in $\mathcal{L}$ into a \texttt{PA}-proof of the same formula. Let a proof of a formula in $\mathcal{L}$ in TB be given. By Lemma \ref{lem:finit-ax-tb} we have that there are finitely many sentences occuring as axioms in the proof. Let  . Hence every axiom in the proof is either an axiom of \texttt{PA} or an instance of the induction schema. Let the formula $\tau(x)$ be the formula: 
\begin{align*}
    (x = \ulcorner \varphi_1 \urcorner \land \varphi_1) \lor ... \lor (x = \ulcorner \varphi_n \urcorner \land \varphi_n).
\end{align*}
Then we replace all occurences of the truth predicate $T$ in the given \texttt{TB}-proof with $\tau$. Then the disquotation ...
\end{proof}
